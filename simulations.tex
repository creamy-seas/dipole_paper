% -*- TeX-master: "../dipole_ilya_paper.tex" -*-
\section{Dipole Transition}
\label{sec:dipole-transition}

We analyse  the tuneability  of the \iket{2}  \ra \iket{1} transition  rate with  reference to
Fermi's golden rule \includeref{some shit}

 \begin{equation}\label{eqn:goldenRule}
   \frac{1}{T_{21}} = \frac{1}{Z_0}\hbar\omega_{21}\frac{\iabsSquared{\bra{2}T\iket{1}}}{\hbar^2}.
 \end{equation}

 \noindent The kinetic  term $ T $ plays  the role of the dipole operator  for this transition
 and  is evaluated  with  the  eigenstates \iket{1},  \iket{2}  of  the Hamiltonian.   Highest
 transition      rates      are      achieved     around      the      degeneracy      biases,
 $ \varphi_{\text{ext}} = (2n+1)\pi, n\in\mathbb{Z} $, see Fig.~\ref{fig:dipole_transition}. This region
 is thus favourable for quick state operations with external driving fields.
 
 \begin{figure}
   \centering \includegraphics[height=6.5cm]{figure4_v1}
 	\caption{The dipole transition rate $ \Gamma $ is proportional to the dipole transition element $ \bra{2}T\ket{1} $, where $ T $ is the charge operator responsible for the transition, and $ \iket{1} $ and \iket{2} are the eigenstates of the system. %There is strong transition between the levels at the turning point in magnetic flux, $ \Phi = (n+\frac{1}{2})\Phi_0, n\in\mathcal{Z} $.
          \label{fig:dipole_transition}}
      \end{figure}

%%% Local Variables:
%%% mode: latex
%%% TeX-master: "../dipole_ilya_paper"
%%% End:
