
% -*- TeX-master: "../dipole_ilya_paper.tex" -*-
\section{Operations with qubits}
\label{sec:characterisation}

% \red{Need scattering data}

\red{Coupling data and comparison of dipole moments between
  RF and twin qubits}

\noindent We record  the energy spectrum of  the twin qubit
while  sweeping the  biasing magnetic  flux.  Because  of a
small asymmetry,  $\eta$, the  fluxes linked through  the left
and right loops are $ \Phi = \frac{\varphi}{2\pi}\Phi_0$ and $ \eta\Phi $.

The \iket{1}~\ilra~\iket{2} transition, $\omega_{21}$, is mapped
with a network analyzer, which measures the transmission of
signal $\omega_{\text{NA}}$  through the  system.  For  the most
part,  the signal  passes through  without any  interaction
with  the  qubit  and  after correcting  for  line  losses,
transmission  is   $  \sim  100\%  $.    Only  near  resonance
($\omega_{\text{NA}}=\omega_{21}$), does  the qubit  exchange photons
with the driving field as it evolves between the ground and
excited  states \cite{rabi}.  This evolution  emits a  wave
that  is  exactly  in  anti-phase with  the  driving  field
\cite{abdumalikov2010},  and the  destructive superposition
in  the output  line  results in  a  transmission dip,  see
Fig.~\ref{fig:transmission}.  The  bottom  of this  dip  is
plotted  with  blue circles  for  different  points in  the
magnetic   flux  to   get   the   transition  spectrum   of
$\omega_{21}$, see inset of Fig.~\ref{fig:transmission}.

\begin{figure}[h]
  \centering            \includegraphics[height           =
  4cm]{transmission_combined.png}
  \caption{\small  \textbf{Mapping   the  qubit  transition
      spectrum.}   For the  lower transition  $\omega_{21}$
    (blue,  inset) a  network analyzer  measures the  power
    transmission coefficient, \iabsSquared{t}, at flux bias
    $ \Phi $ and microwave frequency $ \omega_{21}/2\pi$. A
    Lorentzian fit \cite{Astafiev2010}  to the transmission
    profile  establishes the  resonant frequency,  which is
    marked with blue points on the flux-frequency spectrum.
    For  transition $\omega_{32}$  (red, inset)  a two-tone
    measurement is  run using the  identified $\omega_{21}$
    as  a  weak  probe  .  Readings  are  taken  about  the
    degeneracy point  $ \Phi  \sim \Phi_{0}/2 $,  where the
    low curvature of the applied  flux is stable enough for
    measurements.}
  \label{fig:transmission}
\end{figure}

The  \iket{2}\ilra\iket{3}  transition,  $\omega_{32}$,  is
mapped using two-tone spectroscopy. The network analyzer is
tuned to  the transition  frequency $ \omega_{21}  $, found
with  the first  measurement and  called the  probe signal,
while an  additional generator  sweeps a  second frequency,
$  \omega_{\text{GEN}} $.   Whenever the  generator strikes
the             \iket{2}\ira\iket{3}             transition
($\omega_{\text{GEN}} =  \omega_{32} $), the qubit  will be
undergo     a    ladder     of    excitations,     \iket{1}
\iratext{$\omega_{21}$}\iket{2}     \iratext{$\omega_{32}$}
\iket{3},  depopulating   states  \iket{1}   and  \iket{2}.
Because of this depopulation, the probe signal becomes less
involved with  driving and  it's transmission moves  out of
the  dip in  Fig.~\ref{fig:transmission}.  This  identifies
$\omega_{32}$ which is mapped with red circles.

% prove  that state  1 becomes  depopulated by  solving the
% master equation for two drives


\begin{figure}[h]
  \includegraphics[height=6cm]{figure3_qubit2}
  \caption{\small  Transition  frequencies  between  levels
    \iket{1}\ilra  \iket{2}, $  \omega_{21}  $ (blue),  and
    \iket{2}   \ilra   \iket{3},  $   \omega_{32}$   (red).
    Readings for $ \omega_{32} $ are in a narrow flux range
    because                    away                    from
    $ \Phi = (n  + \frac{1}{2})\Phi_0, n\in\mathbb{Z} $, it
    gets harder to tune the VNA to $ \omega_{21} $ (as part
    of the two-tone  spectroscopy procedure) which prevents
    the accurate mapping of $ \omega_{32} $ with the second
    tone.  Asymmetry  in the flux penetrating  the left and
    right loops results in the gradual change of transition
    frequencies   with  every   $  \Phi_{0}   $  period   -
    $\omega_{21}$  creeps  up, while  $\omega_{32}$  creeps
    down, breaking the usual periodicity in flux qubits.}
  \label{fig:experiment}
\end{figure}

We match  the experimental data points  to simulations made
on  the system's  Hamiltonian, $  \mathcal{H}  = T  + U  $,
developed   using  the   standard   approach  for   quantum
electrodynamics  \cite{orlando1999}.  Islands,  isolated by
the JJ  in Fig.~\ref{fig:setup}, are labeled  with a Cooper
pair  occupation  $ \vec{n}  =  (n_1,  n_2, n_3)  $,  phase
$ \vec{\varphi}  = (\varphi_1, \varphi_2, \varphi_3)  $ and
voltage $ \vec{V} = \left(V_{1}, V_{2}, V_{3}\right) $.  An
inspection  of  the  system's capacitor  system  links  the
charges and voltages

\begin{equation}
  \label{eq:link}
  2e\vec{n} = \hat{C}\vec{V}
\end{equation}

\noindent through the capacitance matrix

\begin{equation}
  \label{eq:capac}
  C = \iabs{C} \begin{pmatrix}
    2  &  -1  &  0\\
    -1  &  2  +  \alpha  &  -1\\
    0  &  -1  & 2
  \end{pmatrix},
\end{equation}

\noindent where  \iabs{C} is  the capacitance of  the outer
JJs.   The   total  charging  energy  resulting   from  the
interaction   between   charge   of   the   Cooper   pairs,
$ \vec{Q}=2e\vec{n}  $, and  voltages on the  islands gives
rise to the kinetic term of the Hamiltonian:

\begin{equation}\label{eq:kinetic}
  \begin{aligned}
    T     &     =     \frac{1}{2}\sum_{i=1}^{3}Q_iV_i     =
    \frac{(2e)^2}{2}\vec{n}\hat{C}^{-1}\vec{n}^{T}.
  \end{aligned}
\end{equation}


Each     of      the     5     JJs      each     contribute
$ E_{Ji}\left(1 - \cos(\varphi_i)\right) $ to the potential
term.    The    two   junction   phases    unspecified   by
$  \vec{\varphi}  $ are  pinned  by  the flux  quantization
condition    for     the    left    and     right    loops,
$  \sum_{i}^{\text{loop}}   \varphi_i  =  2\pi  n,   n  \in
\mathbb{Z}$,   which    have   external    biasing   fluxes
$ \varphi_\text{ext} $, $ \eta\varphi_\text{ext} $:
\begin{equation}\label{eq:potential}
  \begin{aligned}
    U & = E_J\big[4 + \alpha - \alpha\cos(\varphi_{2}) -\cos(\varphi_{1}) -\cos(\varphi_{3}) - \\
    &    \qquad    \cos(\varphi_{2}   -    \varphi_{1}    -
    \varphi_{\text{ext}}) -  \cos(\varphi_{2} - \varphi_{3}
    + \eta\varphi_{\text{ext}})\big].
  \end{aligned}
\end{equation}

The Hamiltonian matrix is encoded in the Cooper pair number
basis, $\vec{n}  $, in which  the kinetic terms are  on the
diagonal   and   the   potential  terms   are   distributed
symmetrically  on the  off diagonal  positions.  The  phase
operators   in  the   number   basis  representation   read
$           e^{\pm            i\hat{\varphi}_j}           =
\sum_{n_i}\iketbra{n_i\pm1}{n_i}$     \cite{phase}.     The
eigenenergies  of the  resulting  Hamiltonian are  compared
with  the  experimental data  in  Fig.~\ref{fig:experiment}
using  \iunit{E_J =  91.0}{GHz}, \iunit{E_C  = 13.50}{GHz},
\iunit{\alpha  =  1.023}{},  \iunit{\eta =  1.011}{}.   The
asymmetry value,  $ \eta $,  is close the visual  loop area
difference   of   3\%   seen   from  the   SEM   image   in
Fig.~\ref{fig:setup}.

The resonance is period in  flux, with a tendency of higher
$\omega_{21}$ at higher magnetic flux numbers.
 
 \begin{figure}[h!]
   \includegraphics[height = 5cm]{figure5}
   \caption{Rabi oscillation taken at the degeneracy point,
     $  \Phi_0/2  $, by  driving  the  qubit with  resonant
     microwaves,  $\omega_{\text{VNA}}  = \omega_{21}$  for
     different  time periods,  $ dt  $, and  monitoring the
     signal  in   the  output  line  \cite{rabi}.    The  a
     decoherence                   time                  of
     $  \tau_{\text{dec}} =  \iunit{42}{ns} $  is extracted
     from the decay envelope,  $ e^{-dt/\tau_\varphi} $, of
     the the oscillations. \label{fig:rabi}}
 \end{figure}
 dfsdf An important qubit parameter is the curvature at the
 turning  points  in  the   energy  spectrum,  where  qubit
 operations are carried out.  A low curvature is desirable,
 to make the qubit less sensitive to external flux changes,
 which would improve decoherence time.  At the twin qubits'
 degeneracy                                           point
 $ \Phi  = (n  + \frac{1}{2})\Phi_0, n\in\mathbb{Z}  $, the
 curvature  is $  -550\pm10\,\text{GHz}/\Phi_0^2 $.   It is
 substantially    smaller    than    $    13\times    10^4$
 $   \text{GHz}/\Phi_0^2$   on    the   4-JJ   flux   qubit
 \cite{stern2014},  $ 8.4  \times 10^4$  \cite{zhu2010} and
 $     37\times      10^{4}$     $     \text{GHz}/\Phi_0^2$
 \cite{gustavsson2012} on the 3-JJ flux qubits demonstrated
 recently.  The decoherence time  in our qubits was however
 relatively                   small,                   only
 $   \tau_\text{dec}   =    \iunit{42}{ns}   $.    We   get
 $\tau_\text{dc}$  from measurement  of Rabi  oscillations,
 see Fig.~\ref{fig:rabi}  \cite{rabi}.  We explain  this by
 poisoning  of  the  sample with  infrared  radiation,  and
 simplified technology used for fabrication as we mentioned
 above.

 Despite the twin qubit have a much the decoherence time in
 our qubits  was relatively small This  improved robustness
 to  flux noise  is  matched  by a  decoherence  time of  ,
 extracted  from Rabi  oscillations in  Fig.~\ref{fig:rabi}
 \cite{rabi}.

%%% Local Variables:
%%% mode: latex
%%% TeX-master: "../dipole_ilya_paper"
%%% End:
