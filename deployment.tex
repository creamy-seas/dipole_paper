% -*- TeX-master: "../dipole_ilya_paper.tex" -*-
\section{Characterisation}
\label{sec:characterisation}

\red{Need scattering  data}

\red{Coupling data  and comparison of  dipole moments
  between RF and twin qubits}

\noindent The energy  spectrum is taken under an external  magnetic field applied
perpendicularly    to    the    plane    of    the    qubit,    linking    fluxes
$ \Phi  = \frac{\varphi}{2\pi}\Phi_0$ and $  \eta\Phi $, where  $ \eta $  is the area ratio  between the
loops. The \iket{1}~\ilra~\iket{2} transition, $\omega_{21}$,  is mapped with a Vector
Network  Analyzer  (VNA)  that  sends  in a  microwave  signal  of  a  particular
frequency,  $\omega_{\text{VNA}}$, and  monitors it's  amplitude after  it has  passed
through  the  system. As  shown  in  previous works  \cite{abdumalikov2010},  the
transmission       amplitude        is       suppressed        at       resonance
($\omega_{\text{VNA}}=\omega_{21}$).  This  resonance condition  is  mapped  out with  blue
circles in Fig.~\ref{fig:experiment}.

The  \iket{2}\ilra\iket{3}   transition,  $\omega_{32}$,  is  mapped   using  two-tone
spectroscopy:  the  VNA  is  tuned to  the  aforementioned  transition  frequency
$ \omega_{21}  $, while an additional  microwave generator sweeps a  second frequency,
$  \omega_{\text{GEN}}  $. Whenever  the  generator  strikes the  \iket{2}\ira\iket{3}
transition ($\omega_{\text{GEN}} = \omega_{32} $), the  qubit will be excited in a sequence
\iket{1}  \iratext{$\omega_{21}$}\iket{2}  \iratext{$\omega_{32}$}  \iket{3},  depopulating
state \iket{1}. This will show up in  an improved transmission of the VNA signal,
$\omega_{21}$,   as  there   is   no  longer   a  state   \iket{1}   to  excite.   The
$\omega_{\text{GEN}}$ for which the transmission of $\omega_{21}$ is boosted is mapped with
red circles.

% prove that state  1 becomes depopulated by solving the  master equation for two
% drives

\begin{figure}[h]
  \includegraphics[height=6cm]{figure3_qubit2}
  \caption{\small  Transition  frequencies  on  the  twin  qubit  between  levels
    \iket{1}\ilra  \iket{2}, $  \omega_{21}  $ (blue),  and  \iket{2} \ilra  \iket{3},
    $ \omega_{32}$ (red).  Readings for $ \omega_{32}  $ are in a narrow flux range because
    away from $ \Phi = (n +  \frac{1}{2})\Phi_0, n\in\mathbb{Z} $, it gets harder to tune
    the VNA to $ \omega_{21} $ (as  part of the two-tone spectroscopy procedure) which
    prevents the accurate mapping of $  \omega_{32} $ with the second tone.  Asymmetry
    in  the flux  penetrating the  left and  right loops  results in  the gradual
    change   of  transition   frequencies  with   every  $   \Phi_{0}  $   period  -
    $\omega_{21}$  creeps  up,   while  $\omega_{32}$  creeps  down,   breaking  the  usual
    periodicity in flux qubits.}
  \label{fig:experiment}
\end{figure}

\noindent  We match  the  experimental data  points to  simulations  made on  the
system's Hamiltonian,  $ \mathcal{H}  = T  + U  $, which  we build  following the
approach in standard quantum electrodynamics problems \cite{orlando1999}. Islands
in   Fig.~\ref{fig:setup}   are   labeled   with   a   Cooper   pair   occupation
$       \vec{n}        =       (n_1,       n_2,       n_3)        $,       phases
$      \vec{\varphi}      =      (\varphi_1,      \varphi_2,      \varphi_3)      $      and      voltage
$ \vec{V}  = \left(V_{1}, V_{2},  V_{3}\right) $.  An inspection of  the system's
topology links the sets

\begin{equation}
  \label{eq:link}
  2e\vec{n} = \hat{C}\vec{V}
\end{equation}

\noindent           through            the           capacitance           matrix
$ C=\iabs{C}\left(\begin{smallmatrix} 2 & -1 & 0\\ -1  & 2 + \alpha & -1\\ 0 & -1
    & 2 \end{smallmatrix}\right)$, where \iabs{C} is the capacitance of the outer
JJs. The total  charging energy resulting from the interaction  between charge of
the Cooper  pairs, $ \vec{Q}=2e\vec{n} $,  and voltages on the  islands generates
the kinetic term of the Hamiltonian:

\begin{equation}\label{eq:kinetic}
  \begin{aligned}
    T           &          =           \frac{1}{2}\sum_{i=1}^{3}Q_iV_i          =
    \frac{(2e)^2}{2}\vec{n}\hat{C}^{-1}\vec{n}^{T}.
  \end{aligned}
\end{equation}

The  5 JJs  each contribute  $ E_{Ji}\left(1  - \cos(\varphi_i)\right)  $ to  the
potential term.  The two  junction phases  unspecified by  $ \vec{\varphi}  $ are
pinned  by  the  flux  quantization  condition for  the  left  and  right  loops,
$  \sum_i^{\text{loop}}  \varphi_i =  2\pi  n,  n  \in \mathbb{Z}}$,  which  have
external biasing fluxes $ \varphi_\text{ext} $, $ \eta\varphi_\text{ext} $:
\begin{equation}\label{eq:potential}
  \begin{aligned}
    U & = E_J\big[4 + \alpha - \alpha\cos(\varphi_{2}) -\cos(\varphi_{1}) -\cos(\varphi_{3}) - \\
    &   \qquad  \cos(\varphi_{2}   -   \varphi_{1}   -  \varphi_{\text{ext}})   -
    \cos(\varphi_{2} - \varphi_{3} + \eta\varphi_{\text{ext}})\big].
  \end{aligned}
\end{equation}

The Hamiltonian matrix is encoded in the Cooper pair number basis, $\vec{n} $, in
which the kinetic  terms align on the  diagonal axis and the  potential terms are
distributed symmetrically on  the off diagonal positions (the  phase operators in
the            number             basis            representation            read
$ e^{\pm i\hat{\varphi}_j} = \sum_{n_i}\iketbra{n_i\pm1}{n_i}$ \cite{phase}). The
eigenenergies of  the resulting  Hamiltonian are  compared with  the experimental
data  in Fig.~\ref{fig:experiment}  using \iunit{E_J  = 91.0}{GHz},  \iunit{E_C =
  13.50}{GHz}, \iunit{\alpha  = 1.023}{},  \iunit{\eta = 1.011}{}.  The asymmetry
value, $ \eta  $, is close the visual  loop area difference of 3\%  seen from the
scanning microscope image (Fig.~\ref{fig:setup}).
 
 \begin{figure}[h!]
   \includegraphics[height = 5cm]{figure5}
   \caption{Rabi oscillation measurements are taken  at the degeneracy flux bias,
     $   \Phi_0/2   $,  by   driving   the   qubit  with   resonant   microwaves,
     $\omega_{\text{VNA}} = \omega_{21}$ for different  time periods, $ dt $, and
     monitoring the signal in the output line \cite{rabi}. The a decoherence time
     of $  T_\varphi =  \iunit{42}{ns} $  is extracted  from the  decay envelope,
     $ e^{-dt/\tau_\varphi} $, of the the oscillations. \label{fig:rabi}}
 \end{figure}

 At the degeneracy flux bias $  \Phi = (n + \frac{1}{2})\Phi_0, n\in\mathbb{Z} $,
 the energy levels  posses a low curvature of  $ -550\pm10\,\text{GHz}/\Phi_0^2 $
 compared  with   $  13\times   10^4$  \cite{stern2014},   $  8.4   \times  10^4$
 \cite{zhu2010}     and     $      37\times     10^{4}$     \cite{gustavsson2012}
 $  \text{GHz}/\Phi_0^2$  demonstrated  recently  on  other  architectures.  This
 improved  robustness  to  flux  noise  is  matched  by  a  decoherence  time  of
 $  \tau_\phi   =  \iunit{42}{ns}   $,  extracted   from  Rabi   oscillations  in
 Fig.~\ref{fig:rabi} \cite{rabi}.  Given that the  qubit was not  fabricated with
 the technological advancements discussed above, this  is a good initial value to
 start work on.

%%% Local Variables:
%%% mode: latex
%%% TeX-master: "../dipole_ilya_paper"
%%% End:
