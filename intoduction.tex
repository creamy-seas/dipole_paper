\section{Introduction}
 \noindent Superconducting qubits are in the running for the devices of choice in scalable qunatum electronic platforms, paralleling the functionality of a transistor \cite{Astafiev2010}\cite{hoi2011}, multiplexer \cite{honigl2018} and serial bus \cite{shen2005}. They have a range of tuneable parameters, to fit the use-cases of different environments, can be built with industrial fabrication techniques and integrated at scale \cite{johnson2010}. 
 
 That said, superconducting qubits are still victims of short coherence times: the transmon qubits in the revolutinary 5-qubit quantum experience from IBM $ \sim 60\,\mu $s (\href{http://www.research.ibm.com/ibm-q}{research.ibm.com/ibm-q}) \cite{linke2017}, compared to their trapped cavity counterparts $\sim1\, $ms \cite{monroe1995}. Such decoherence is a result of the large capacitive elements in these macroscopic structures ($>1\,\mu\text{m}$) coupling strongly to charge variations in the external environment. This fluctuates the qubit's energy levels, leading to erratic evolution of the quantum state \cite{devoret2008}. For a sufficient number of quantum logic operations pertaining to long computations ($ 10^{4} $) \cite{orlando1999}, coherence times need to surpass the $ 100\,\mu$s barrier in superconducting systems.
 
 Flux qubit architectures \cite{chiorescu2003}\cite{mooij1999} were developed to address this problem by making Josephson junction energies dominant over the charging energies, lowering the devices charge sensitivity \cite{orlando1999}. Branching of architectures in this direction has lead to improved coherence times: quantronium $\sim500\,$ns \cite{cottet2002} \cite{gu2017}, shunted phase qubit $\sim10\,\mu $s \cite{stern2014} , shunted flux qubit $\sim80\,\mu$s \cite{yan2016} , 4-JJ \cite{qui2016}, fluxonium $\sim1\,$ms \cite{pop2014}. 
 
 Appending to this list is a new `twin' qubit, consisting of two 3-JJ qubits, linked by a common $ \alpha-$JJ (Fig~\ref{fig:setup}). A chain from 15 such qubits was recently placed into a coplanar waveguide that demonstrated flux-tunable transmission \cite{shulga2018}. Results showed a weak flux dependence of the transition energy when the chain was flux biased to the degeneracy point, pointing to an operational regime of a superconducting qubit with both low flux and charge sensitivities.
 
 In this work we isolate one of these twin qubits and provide experimental and theoretical evidence for: strong anharmonicity of the system with respect to the \iket{1}\lra\iket{2} and \iket{2}\lra\iket{3} transitions; weak flux dependence of the transition energies at the degeneracy bias $\sim \frac{1}{2}\Phi_0 = \frac{h}{4e}$, with a decoherence rate $ \tau_\text{decoherence} = \frac{2\pi}{\Gamma}= \red{extract!} $; tuneable capacitive coupling to the tranmission line, resulting from a flux-tuneable dipole moment.
 
 \includeref{Phase is not localised, due to infinite cosine potential.}