% -*- TeX-master: "../dipole_ilya_paper.tex" -*-

\section{Decoherence results in loss of quantum information}
\label{sec:decoh-results-loss}

\noindent Quantum  processing involves manipulating  the state of  a qubit, $\Psi$,  changing the
relative state population, $\ensuremath{|\alpha|} \le 1$, and phase, $\varphi$, between states \iket{0} and
\iket{1}:
\begin{equation}
  \label{eq:4}
  \Psi = \alpha\iket{0} + e^{i\varphi}(1-\alpha)\iket{1}.
\end{equation}

\noindent  When  written as  a  density  matrix, the  phase  information  is mapped  onto  the
off-diagonal elements:
\begin{equation}
  \label{eq:5}
  \rho = \iketbra{\Psi}{\Psi} = \begin{pmatrix}
    \iabsSquared{\alpha}  & \alpha(1-\alpha)e^{-i\varphi}\\
    \alpha(1-\alpha)e^{+i\varphi} & \iabsSquared{(1-\alpha)}.
  \end{pmatrix}
\end{equation}

\noindent Decoherence,  by definition, causes  the off-diagonal elements  decay to 1/e  of the
initial  value over  a time  $\tau_{\text{dec}}$. Decoherence  causes the  loss of  computational
information encoded in the phase $\varphi$.

\section{Representation of Hamiltonian in the charge basis}
\label{sec:repr-hamilt-charge}

\noindent The Hamiltonian

\begin{equation}
  \label{eq:hamitlian_revisited}
  \begin{aligned}
    \mathcal{H} & = T + U\\
    & = E_C \iabs{C} \iaverage{\hat{C}^{-1}}_{\iket{n_{1}, n_2, n_3}} \\
    & \quad+  E_J\big[4 + \alpha - \alpha\cos(\varphi_{2}) -\cos(\varphi_{1}) -\cos(\varphi_{3}) - \\
    &  \qquad  \cos(\varphi_{2}   -  \varphi_{1}  -  \varphi_{\text{ext}})  -  \cos(\varphi_{2}   -  \varphi_{3}  +
    \eta\varphi_{\text{ext}})\big]
  \end{aligned}
\end{equation}

\noindent in the charge basis  takes the  form  shown in  Fig.~\ref{fig:matrix_representation}, where a state \iket{-1, 0,  1} would correspond to a CP-occupation of -1, 0 and 1 on islands 1, 2 and 3.

Kinetic terms  ($ {T}  $) naturally  fall on  the diagonal  axis of  the matrix.
Potential terms ($  U $) are expanded to complex  exponentials, which become off
diagonal elements (see Appendix~\ref{sec:expans-potent-term}).

To decide on the number of states for the simulation, we took a sufficiently complete system state of 19 interacting CPs, and methodically ``switched off'' interactions between the high-CP-number states. I logged the deviation of the energy spectra of Fig.~\ref{fig:matrix_representation}, where it shows that 9-CP-states describes the system almost as well as with the complete system state.

\begin{figure}[h]
  \centering\def\svgwidth{8cm}\import{images_inkscape/}{fig4.pdf_tex}
  \caption{\small \textbf{Hamiltonian in the CP-basis representation for 3-CP-states-per-island (27 system states):} Purple square denote the kinetic terms that all fall on the main diagonal. Light blue squares denote simple off-diagonal terms distributed symmetrically about the main diagonal, arising from e.g. $\cos(\varphi_2)$. Dark blue squares are have an additional flux dependence $e^{i\varphi_{\text{ext}}}, e^{i\eta\varphi_{\text{ext}}}$, arising from e.g. $\cos(\varphi_2-\varphi_1-\varphi_{\text{ext}})$.
    \label{fig:matrix_representation} }
\end{figure}

\begin{figure}[h]
  \centering\def\svgwidth{8cm}\import{images_inkscape/}{fig7.pdf_tex}
  \caption{\small \textbf{Choosing the lowest number of interacting CP, that would  capture the nature of perturbation effects:} In the limiting case of too few CP-interactions (potential terms \,$U$\,), the full system states (19-CP) is substantially deviated from. At the sane tine, it made little sense to include more than 9 CP in the simulation.
    \label{fig:matrix_representation} }
\end{figure}

% For        example         $        \cos(\varphi_{2})         $        becomes
% $    \mathbb{I}_{1}\otimes\frac{1}{2}\left[\sum_{n_2}\iketbra{n_2+1}{n_2}    +
%   \iketbra{n_2-1}{n_2}   \right]\otimes\mathbb{I}_{3}    $,   where   identity
% operators $ \mathbb{I}_{1,3} $ carry the states of the non-involved islands.


\section{Expansion of potential term, $U$, in the CP-basis\label{sec:expans-potent-term}}

\noindent Switching to the CP-basis, leads  to a non-trivial representation of phase-dependent
terms in  $ U(\varphi_1,\varphi_2,\varphi_3,\varphi_{\text{ext}})  $.  The following  explanation will  illustrate the
expansion process.

\begin{enumerate}
\item Derive  the commutation relation  between the number, $  \hat{n} $, and  the exponential
  phase,     $    e^{\pm     i\hat{\varphi}}     $,    by     using     the    standard     relation
  $ \left[\hat{n},\hat{\varphi}\right] = 1 $:
 
  {\scriptsize\begin{equation}\label{eq:ab1}
      \begin{aligned}
        \icommutation{\hat{\red{n}}}{e^{\pm i \hat{\blue{\varphi}}}} & =  \icommutation{\hat{n}}{\sum_{\alpha = 0}^{\infty} \frac{(\pm i\hat{\blue{\varphi}})^\alpha}{\alpha!}} = \sum_{\alpha = 0}^{\infty} (\pm i)^\alpha\frac{\icommutation{\hat{\red{n}}}{\hat{\blue{\varphi}}^\alpha}}{\alpha!}\\
        &  = \sum_{\alpha  = 0}^{\infty}  (\pm i)^\alpha\frac{-\alpha  i\hat{\blue{\varphi}}^{\alpha -  1}}{\alpha!}  =  \pm \sum_{\alpha  =
          1}^{\infty}i^{\alpha  -  1}\frac{(\pm\hat{\blue{\varphi}})^{\alpha  -  1}}{(\alpha   -  1)!}   =  \pm  e^{\pm
          i\hat{\blue{\varphi}}}.
      \end{aligned}
    \end{equation}}

\item Operating with the number operator on state $ e^{\pm i\hat{\blue{\varphi}}}\ket{n} $ and using
  the commutation result \begin{equation}\label{eq:ab2}
    \begin{aligned}
      \hat{\red{n}}\bigg[e^{\pm   i\hat{\blue{\varphi}}}\ket{n}\bigg]    =   &    \bigg[\pm   e^{\pm
        i\hat{\blue{\varphi}}}   +  e^{\pm   i\hat{\blue{\varphi}}}\hat{n}\bigg]\ket{n}  \\   &  =   (n\pm
      1)\bigg[e^{\pm i\hat{\blue{\varphi}}}\ket{n}\bigg].
    \end{aligned}
  \end{equation}

\item Evidently, the exponential phase operator is a ladder operator for the \iket{n} state:
  \begin{equation}
    \label{eq:ab5}
    e^{\pm i\hat{\blue{\varphi}}}\ket{n} = \ket{n \pm 1} \Rightarrow e^{\pm i\blue{\varphi}} = \sum_{{n}} \iketbra{n\pm1}{n}.
  \end{equation}
 
\item Thus  operator $\cos(\hat{\varphi}_2-\hat{\varphi}_1-\hat{\varphi}_{\text{ext}})$ can be  expressed in the
  number basis $ \left\{n_{1},n_2,n_3\right\} $ as: {\scriptsize
    \begin{equation}
      \label{eq:ab4}
      \begin{aligned}
	\cos(\hat{\varphi}_2-& \hat{\varphi}_1-\hat{\varphi}_{\text{ext}}) =\\
        & = \frac{1}{2}\left(e^{i\hat{\varphi}_{2}}e^{-i\hat{\varphi}_1}e^{-i\hat{\varphi}_{\text{ext}}}+\text{c.c.}\right)\\
        & =  \frac{1}{2} \left(\left[\sum_{n_2}\iketbra{n_2+1}{n_2}\right]\right. \\
        & \qquad \left.\otimes \left[\sum_{n_1}\iketbra{n_1-1}{n_1}\right]
          \otimes\mathbb{I}^{(3)}\right)e^{-i\varphi_{\text{ext}}} + \text{c.c.}\\
        &             =              \frac{1}{2}e^{-i\varphi_{\text{ext}}}             \sum_{n_{1,2,3}}
        \iketbra{n_1-1,n_2+1,n_3}{n_1,n_2,n_3} + \text{c.c.}
      \end{aligned}
    \end{equation}}
  
\item Physically this corresponds  to a CP exchange between island 1 and  island 2. An example
  of a term could be
  \begin{equation}
    \label{eq:ab6}
    \frac{1}{2}e^{-i\varphi_{\text{ext}}}\iketbra{-1,1,0}{0,0,0} + \frac{1}{2}e^{+i\varphi_{\text{ext}}}\iketbra{0,0,0}{-1,1,0},
  \end{equation}
  \noindent  which would  be a  pair of  symmetrical off-diagonal  elements in  the matrix  of
  Fig.~\ref{fig:matrix_representation}.
\end{enumerate}


% \section{Rabi oscillation measurements}
% \label{sec:rabi-oscill-meas}

% \noindent Rabi  oscillations are observed when  a qubit system, with  a transition frequency
% $\omega_{21}$, is driven by microwaves in  resonance with this transition.  The expectation value
% of  $ \iaverage{\sigma_{-}}  $, $\sigma_{-}=\iketbra{0}{1}$,  which is  measured by  a vector  network
% analyzer \cite{Astafiev2010}, oscillates sinusoidally with the  length of the drive as shown
% below.

% \begin{enumerate}
% \item The Hamiltonian during microwave driving of the system is a combination of the qubit's
%   two-level system Hamiltonian
%   \begin{equation}
%     \label{eq:rabi1}
%     \mathcal{H}_{q} = -\frac{\hbar\omega_{21}}{2}\sigma_{z},
%   \end{equation}

%   \noindent                                                                            where
%   $\sigma_z = \ensuremath{\left(\begin{smallmatrix} 1  & 0\\0 & -1 \end{smallmatrix}\right)}
%   $, and the Hamiltonian of the resonant drive of strength $ \hbar\Omega $
%   \begin{equation}
%     \label{eq:rabi2}
%     \mathcal{H}_{d} = \hbar\Omega\cos(\omega_{21}t)\sigma_{x}
%   \end{equation}

%   \noindent   that    couples   the   two    levels   through   the    transition   operator
%   $\sigma_x = \ensuremath{\left(\begin{smallmatrix} 0 & 1 \\ 1 & 0 \end{smallmatrix}\right)}
%   $.

% \item              Applying              the             unitary              transformation
%   $     U(t)    =     \exp    \left(-i     \frac{\omega_{21}t}{2}\sigma_z\right)    $     to
%   $ \mathcal{H} = \mathcal{H}_{q}+\mathcal{H}_{d} $ evaluates to
%   \begin{equation}
%     \label{eq:rabi3}
%     \begin{aligned}
%       \mathcal{H}'& = U\mathcal{H}U^{\dag} - i\hbar U\dot{U}^{\dag}\\
%       & \approx \frac{\hbar\Omega}{2}\sigma_x
%     \end{aligned}
%   \end{equation}

%   \noindent    under     the    approximation     that    non-conserving     energy    terms
%   $ e^{\pm 2i\omega_{21}t} $ are neglected.
  
% \item  The evolution  of  a  ground state  in  this  rotated frame  for  a  drive of  length
%   $ \Delta t $ is
%   \begin{equation}
%     \label{eq:rabi4}
%     \begin{aligned}
%       U'(\Delta t)\iket{0} & = e^{-i\frac{\mathcal{H'}}{\hbar}\Delta t}\iket{0} \\
%       &     =      \cos\left({\Omega\Delta     t}/{2}\right)\ket{0}+i\sin\left({\Omega\Delta
%         t}/{2}\right)\ket{1},
%     \end{aligned}
%   \end{equation}

%   \noindent which as a density matrix reads

%   \begin{equation}
%     \label{eq:rabi6}
%     \rho_{\Delta t} = \frac{1}{2}\begin{pmatrix}
%       1 + \cos(\Omega\Delta t) & -i\sin(\Omega\Delta t)\\
%       i\sin(\Omega\Delta t) & 1 - \cos(\Omega\Delta t)
%     \end{pmatrix}
%   \end{equation}

%   \noindent

% \item Rabi oscillations, proportional, to $ \iaverage{\sigma_{-}} $:
%   \begin{equation}
%     \label{eq:rabi5}
%     \begin{aligned}
%       \iaverage{\sigma_{-}} & = \bra{0}U'^{\dag}(t) \iketbra{0}{1} U'(t)\iket{0}\\
%       & = \sin \left(\Omega t\right)
%     \end{aligned}
%   \end{equation}

% \end{enumerate}
