% -*- TeX-master: "../dipole_ilya_paper.tex" -*-
%%%%%%%%%%%%%%%%%%%%%%%%%%%%%%%%%%%%%%%%%%%%%%%%%%%%%%%%%%%%%%%%%%%%%%%%%%%%%%%%%%%% 
% kbordermatrix MUSE  BE PLACED IN /usr/local/texlive/2016/texmf-dist/tex/latex/base  and then
% run sudo -s texhash to load it up
%%%%%%%%%%%%%%%%%%%%%%%%%%%%%%%%%%%%%%%%%%%%%%%%%%%%%%%%%%%%%%%%%%%%%%%%%%%%%%%%%%%%

\documentclass[%
reprint,
% superscriptaddress,   groupedaddress,  unsortedaddress,   runinaddress,  frontmatterverbose,
% preprint, showpacs, preprintnumbers, nofootinbib, nobibnotes, bibnotes,
amsmath, amssymb, pra,
% aps, pra, prb, rmp, prstab, prstper, floatfix
]{revtex4-1}


\usepackage{cmap} %make pdf searchable

\usepackage[toc,page]{appendix} %have an appendix

%%% Geometry
\usepackage{geometry}

\geometry{top=25mm}

\geometry{bottom=30mm}

\geometry{left=30mm}

\geometry{right=20mm}

%%% Linespacing
\usepackage{setspace}

%%% Specific packages
\usepackage{braket} %for the \Bra and \Ket
\usepackage{kbordermatrix} % label matrix rows and columns
\usepackage{transparent} %for transparent options
\usepackage{etoolbox} %logical operators
\usepackage{mathdots} %for iddots <--------------------

%%% Common packages
\usepackage{graphicx} %for images
% \usepackage{caption} \usepackage[makeroom]{cancel}
\usepackage{amsmath,amsfonts,amssymb,amsthm,mathtools} %ams package, for fonts, symbols
\usepackage{comma} %smart comma spacing in formulas e.g. $0,3$ is number $0, 3$ list
\usepackage{url} %addition of url in bibliogrpahy
\usepackage{hyperref} \usepackage{multirow} %merge rows
\usepackage{framed} %adds a frame box
\usepackage{color} %colour
\usepackage{lastpage} %total pages in document
\relpenalty=9999 %prevent splitting of equations
\binoppenalty=9999 \mathtoolsset{showonlyrefs=true} %only label formulas with refferences
\usepackage{bm} % bold math

%%% Importing other files
\usepackage{import}

%%% Hyperreferences
% \usepackage[usenames,dvipsnames,svgnames,table,rgb]{xcolor}
\hypersetup{   unicode=true,   pdftitle={Primary   insights    into   the   `dipole'   qubit},
  pdfauthor={Ilya      Antonov},      pdfsubject={Qubits},     pdfcreator={Ilya      Antonov},
  pdfkeywords={Qubits} {} {}, colorlinks=true, % false: ссылки в рамках; true: цветные ссылки
  linkcolor=black, %internal links
  citecolor=blue, %bib links
  filecolor=magenta, %file links
  urlcolor=cyan %url links
}

%% Images with Inkscape (they are used by it)
\usepackage{xifthen}

\usepackage{transparent}
