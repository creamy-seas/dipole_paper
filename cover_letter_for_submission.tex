\documentclass[%
% reprint,
superscriptaddress,
% groupedaddress,
% unsortedaddress,
% runinaddress,
% frontmatterverbose,
preprint,
preprintnumbers,
% nofootinbib,
% nobibnotes,
bibnotes,
amsmath,
amssymb,
aps,
showkeys,
% pra,
prb,
% rmp, prstab, prstper, floatfix,
]{revtex4-2}

\usepackage{braket}
\usepackage{graphicx}
\graphicspath{{images_inkscape/}}
\usepackage{dcolumn}                    % Align table columns on decimal point
\usepackage{bm}                         % bold mathb
\usepackage{hyperref}                   % add hypertext capabilities
\usepackage[mathlines]{lineno}          % Enable numbering of text and display math
% \linenumbers\relax                    % Commence numbering lines

% \usepackage[showframe,%Uncomment any one of the following lines to test
% scale=0.7, marginratio={1:1, 2:3}, ignoreall,% default settings
% text={7in,10in},centering,
% margin=1.5in,      total={6.5in,8.75in},      top=1.2in,      left=0.9in,      includefoot,
% height=10in,a5paper,hmargin={3cm,0.8in},
% ]{geometry}

%% Common Packages
\usepackage[usenames,dvipsnames,svgnames,table,rgb]{xcolor} % for colouring \color{red!20}
\usepackage{framed}             % \begin{framed} for framed boxes
\usepackage{multirow}               % merge rows in tables

%%% COLOURS
\newcommand{\red}[1]{{\color{red}{#1}}}
\newcommand{\blue}[1]{\textcolor{blue}{#1}}
\newcommand{\green}[1]{\textcolor{green}{#1}}
\newcommand{\purple}[1]{\textcolor{purple}{#1}}
\newcommand{\grey}[1]{{\color{gray!60} #1}}
\newcommand{\ec}{ }                             % end colour command for emacs regexp
\definecolor{amber}{rgb}{1.0, 0.75, 0.0}
\newcommand{\gold}[1]{{\color{amber}{#1}}}

%%% Physics
% Bra and Ket
\newcommand{\iket}[1]{\ensuremath{\Ket{#1}}}
\newcommand{\ibra}[1]{\ensuremath{\Bra{#1}}}
\newcommand{\iketbra}[2]{\ket{#1}\bra{#2}}
\newcommand{\iup}{\ensuremath{\Ket{\uparrow}}}
\newcommand{\idown}{\ensuremath{\Ket{\downarrow}}}
\newcommand{\iupBra}{\ensuremath{\Bra{\uparrow}}}
\newcommand{\idownBra}{\ensuremath{\Bra{\downarrow}}}
\newcommand{\iupKetBra}{\ensuremath{\Ket{\uparrow}\Bra{\uparrow}}}
\newcommand{\idownKetBra}{\ensuremath{\Ket{\downarrow}\Bra{\downarrow}}}
% matrices
\newcommand{\iz}{\ensuremath{\begin{pmatrix}1&0\\0&-1\end{pmatrix}}}
\newcommand{\ix}{\ensuremath{\begin{pmatrix}0&1\\1&0\end{pmatrix}}}
\newcommand{\iy}{\ensuremath{\begin{pmatrix}0&-i\\i&0\end{pmatrix}}}
\newcommand{\idensity}{\ensuremath{\begin{pmatrix}{\rho_{00}} & {\rho_{01}}\\{\rho_{10}}&{\rho_{11}}\end{pmatrix}}}

% symbols
\newcommand{\isigma}{\ensuremath{\vec{\iaverage{\sigma}}}}
\newcommand{\isigmax}{\ensuremath{{\iaverage{\sigma_x}}}}
\newcommand{\isigmay}{\ensuremath{{\iaverage{\sigma_y}}}}
\newcommand{\isigmaz}{\ensuremath{{\iaverage{\sigma_z}}}}
\newcommand{\iadagger}{\ensuremath{a^{\dagger}}}
\newcommand{\isigmaplus}{\ensuremath{\iaverage{\sigma^{+}}}}
\newcommand{\isigmaminus}{\ensuremath{\iaverage{\sigma^{-}}}}
\newcommand{\isigmaplusminus}{\ensuremath{\iaverage{\sigma^{\pm}}}}


%%% Math simplicity
\newcommand{\iunit}[2]{\ensuremath{#1\,\text{#2}}}
\newcommand{\iunitMixed}[3]{\ensuremath{#1\,#2\text{#3}}}
\newcommand{\iabsSquared}[1]{\ensuremath{\left|#1\right|^2}}
\newcommand{\iabs}[1]{\ensuremath{\left|#1\right|}}
\newcommand{\icommutation}[2]{\ensuremath{\left[#1,                                #2\right]}}
\newcommand{\iaverage}[1]{\ensuremath{\left\langle #1 \right\rangle}}
\newcommand{\iderivative}[2]{%
  \ensuremath{%
    \frac{\partial#1}{\partial#2}}}
\newcommand{\ira}{\ensuremath{\,\rightarrow\,}}
\newcommand{\iRa}{\ensuremath{\qquad\Rightarrow\qquad}}
\newcommand{\ilra}{\ensuremath{\,\leftrightarrow\,}}
\newcommand{\iratext}[1]{\ensuremath{\,\xrightarrow{\text{#1}}\,}}

\begin{document}

\begin{flushleft}
  The Editor\\
  Physics Review B - Rapid Communications\\
  \today
\end{flushleft}

\title{Cover letter for submission}

\maketitle






\noindent We are submitting to \texttt{Physics Review B - Rapid Communications} the paper
\begin{center}
  \textbf{The superconducting twin qubit}\\
  \emph{I.  V.  Antonov}, \emph{R. S.  Shaikhaidarov}, \emph{V.  N.  Antonov}, \emph{O.V.  Astafiev}
\end{center}


\noindent In our work we study modification of flux qubit, which differs from the standard flux qubit geometry and is formed by two loops. The symmetric layout provides some interesting features. The experimentally studied qubit has high anharmonicity and low sensitivity to global flux noise. The qubit is capacitively coupled to a transmission line, which allows to characterise its spectrum. We simulate the qubit energy spectrum and extract its parameters with a standard quantum circuit model. The model nicely fits the experimental results. The novel qubit and analysis forms a solid platform for further development of the flux qubit family towards application to quantum processor. We provide experimental evidence of its transition spectrum and decoherence behaviour as well as outlining the fabrication and measurement procedure.

The work will be of interest to a wide scientific community working in the field of quantum superconducting circuitry. Also it is an interesting system to explore from the view of fundamental quantum physics.

\vspace{1em}
\begin{flushleft}
  Corresponding author is:\\
\textbf{Ilya Antonov}\\
Physics Dept, RHUL Royal Holloway University of London\\
Egham, Surrey\\
TW20 0EX, United Kingdom\\
e-mail: ilya.antonov.2013@live.rhul.ac.uk\\
\end{flushleft}

\end{document}
