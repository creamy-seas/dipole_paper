% -*- TeX-master: "../dipole_ilya_paper.tex" -*-

\section{Introduction}
% A  quantum   electronic  platform   that  parallels  the   functionality  of   a  transistor
% \cite{Astafiev2010}\cite{hoi2011},    multiplexer   \cite{honigl2018}    and   serial    bus
% \cite{shen2005} will server  as an intergral part of commercialisation  of quantum computing
% power.  Superconducting quibts are one of the device sof choice

\noindent Superconducting  qubits are  one of  the promising  trends for  implementing quantum
computing technology. Being nothing more than strips of aluminum on a chip, their geometry can
be designed to select  an operating energy, state transition rates  and sensitivity that would
fit the use-case of  very specific environments.  Over the past decade  they have been carried
out the functionality  of a transistor \cite{Astafiev2010}\cite{hoi2011} (a  control field was
used to pass or  block a second field at a  different frequency) multiplexer \cite{honigl2018}
(2 input signals can be mixed on an  artificial atom to controllably generates a single output
signal) and serial bus \cite{shen2005}.  Superconducting qubits can be produced using industry
standard  fabrication  techniques  and  integrated  at  scale  into  large  coherent  circuits
\cite{johnson2010}.   All speaks  to  a strong  case of  servicing  future quantum  electronic
platforms with this technology.

One of the inherent limitation superconducting  qubits face is a comparatively short coherence
time, $\tau_{\text{dec}}$.  It  the time over which  the $\rho_{01}$ and $\rho_{10}$  components of the
system's                                    density                                    matrix,
$\rho    =    \ensuremath{\left(\begin{smallmatrix}    \rho_{00}    &    \rho_{01}    \\    \rho_{10}    &
      \rho_{11} \end{smallmatrix}\right)} $, decay to  zero and computational information becomes
lost \cite{phaseExp}.  For  example, the transmon qubits in the  revolutionary 5-qubit quantum
experience from IBM  (\href{http://www.research.ibm.com/ibm-q}{research.ibm.com/ibm-q}) have a
coherence  time  of  $  \sim  60\,\mu  $s  \cite{linke2017}.   For  reference,  qubits  on  trapped
$ ^{+} $Be$ ^{9} $ ions had a coherence time of $\sim1\, $ms over 20 years ago \cite{monroe1995},
and to  house a sufficient  number of quantum  logic operations for  multi-stage computations,
$ \sim 10^4 $, coherence times need to surpass the $ 100\,\mu$s barrier \cite{orlando1999}.

Strong  decoherence in  superconducting  qubits is  a consequence  of  the large  capacitances
inherent   to    their   geometry,   the   loop    of   the   flux   qubits    are   typically
$           \iunitMixed{1}{\mu}{m}$            or           greater            in           size
\cite{Astafiev2010}\cite{hoi2011}\cite{johnson2010}, which  couples them to  charge variations
in the  external environment.   The charge  fluctuations in the  environment result  in random
changes of  the qubit's energy levels,  leading to an  erratic evolution of the  quantum state
that  `averages'  out  information  on  the   off-diagonal  elements  of  the  density  matrix
\cite{devoret2008}.
 
Flux qubit architectures have been developed to address this decoherence problem by making the
energy of their Josephson junctions (JJ), dominate over the charging energy, E$_J/$E$_C >> 1$,
which  lowered   the  device's  charge  sensitivity   \cite{orlando1999}  \cite{chiorescu2003}
\cite{mooij1999}.  A whole family of flux qubit designs have lead to improved coherence times:
quantronium  $\sim500\,$ns  \cite{cottet2002}  \cite{gu2017},  shunted  phase  qubit  $\sim10\,\mu  $s
\cite{stern2014}  ,  shunted  flux  qubit  $\sim80\,\mu$s  \cite{yan2016}  ,  4-JJ  \cite{qui2016},
fluxonium $\sim1\,$ms \cite{pop2014}.
 
We extended this list  with a `twin' qubit, consisting of two  symmetrical flux qubits, linked
by a  common $ \alpha-$Josephson  Junction (Fig.~\ref{fig:setup}).  A chain  of 15 such  qubits was
recently  placed  into  a  coplanar  waveguide to  demonstrate  flux-tunable  transmission  of
microwaves \cite{shulga2018}.   Of particular interest to  us was the weak  flux dependence of
the systems transition  energy when it was  biased to the degeneracy  point $\frac{1}{2}\Phi_0 $,
making it benefit from both low flux and charge sensitivities.
 
In this work we isolate one of these twin qubits and provide experimental evidence for: strong
anharmonicity with respect to the \iket{1}\ilra\iket{2} and \iket{2}\ilra\iket{3} transitions;
weak   flux    dependence   of    the   transition   energies    at   the    degeneracy   bias
$\sim \frac{1}{2}\Phi_0 =  \frac{h}{4e}$; compliance with simulations of  the energy structure
and controllability of the \iket{1}~\ilra~\iket{2} transition rate.

% tuneable capacitive coupling to the tranmission  line, resulting from a flux-tuneable dipole
% moment.
