% -*- TeX-master: "../dipole_ilya_paper.tex" -*-

\section{Introduction}
% A quantum electronic platform that parallels the functionality of a transistor
% \cite{Astafiev2010}\cite{hoi2011}, multiplexer \cite{honigl2018} and serial bus
% \cite{shen2005} will server as an intergral part of commercialisation of quantum computing
% power. Superconducting quibts are one of the device sof choice
 
\noindent Over the past decade superconducting qubits have successfully paralleled the
functionality of a transistor \cite{Astafiev2010}\cite{hoi2011}, multiplexer \cite{honigl2018}
and serial bus \cite{shen2005}. Being nothing more than strips of aluminum on a chip, their
geometry can be designed to select an operating energy, state transition rates and
sensitivity that would fit the use-case of very specific environments. They can then be mass
produced using industry standard fabrication techniques and integrated at scale into large
coherent circuits \cite{johnson2010}. This all speaks to their strong case of servicing
future quantum electronic platforms.


One of the inherent limitations superconducting qubits still face are their comparatively short coherence times, $\tau_{\text{dec}}$. Over this period the $\rho_{01}$ and $\rho_{10}$ components of the system's density matrix, $\rho = \ensuremath{\left(\begin{smallmatrix} \rho_{00} & \rho_{01} \\ \rho_{10} & \rho_{11} \end{smallmatrix}\right)} $, decay to zero and computational information becomes lost \cite{phaseExp}.
For example, the transmon qubits in the revolutionary 5-qubit quantum experience from IBM (\href{http://www.research.ibm.com/ibm-q}{research.ibm.com/ibm-q}) have a coherence time of $ \sim 60\,\mu $s \cite{linke2017}. In contrast, 20 years ago qubits on trapped $ ^{+} $Be$ ^{9} $ ions already had a coherence time of $\sim1\, $ms \cite{monroe1995}.
Such strong decoherence in superconducting qubits is a consequence of the large capacitances of their macroscopic sized structures (loop features can often be $ 1\iunitMixed{\mu}{m}$ or greater in size \cite{Astafiev2010}\cite{hoi2011}\cite{johnson2010}), coupling strongly to charge variations in the external environment.
This fluctuates the qubit's energy levels, leading to an erratic evolution of the quantum state that `averages' out information on the off-diagonal elements \cite{devoret2008}. To house a sufficient number of quantum logic operations for multi-stage computations ($ 10^4 $) coherence times need to surpass the $ 100\,\mu$s barrier
\cite{orlando1999}.
 
Flux qubit architectures were developed to address this decoherence problem by making the energy of their Josephson junctions (JJ) dominant in the system, which lowered the device's charge sensitivity purely by nature of suppressing the relative size of any charge-related energy fluctuations  \cite{orlando1999}\cite{chiorescu2003}\cite{mooij1999}. The further branching of flux qubit architectures has lead to improved coherence times: quantronium $\sim500\,$ns \cite{cottet2002} \cite{gu2017}, shunted phase qubit $\sim10\,\mu $s \cite{stern2014} , shunted flux qubit $\sim80\,\mu$s \cite{yan2016} , 4-JJ \cite{qui2016}, fluxonium $\sim1\,$ms \cite{pop2014}.
 
This list is extended with a new `twin' qubit, consisting of two symmetrical flux qubits, linked by a common $ \alpha-$JJ (Fig.~\ref{fig:setup}). A chain of 15 such qubits was recently placed into a coplanar
waveguide to demonstrate flux-tunable transmission of microwaves \cite{shulga2018}. Of particular interest to us was the weak flux dependence of the chain's transition energy when it was biased to the degeneracy point $\frac{1}{2}\Phi_0 $ pointing to an operational regime which would benefit from both low flux and charge sensitivities.
 
In this work we isolate one of these twin qubits and provide experimental evidence for: strong anharmonicity of the system with respect to the \iket{1}\ilra\iket{2} and
\iket{2}\ilra\iket{3} transitions; weak flux dependence of the transition energies at the
degeneracy bias $\sim \frac{1}{2}\Phi_0 = \frac{h}{4e}$, with a decoherence rate
$ \tau_\text{decoherence} = 42\, \text{ns} $; simulation of the \iket{1}~\ilra~\iket{2} transition rate.

% tuneable capacitive coupling to the tranmission line, resulting from a flux-tuneable dipole moment.
