% -*- TeX-master: "../dipole_ilya_paper.tex" -*-

\section{Introduction}
% A  quantum   electronic  platform   that  parallels  the   functionality  of   a  transistor
% \cite{Astafiev2010}\cite{hoi2011},    multiplexer   \cite{honigl2018}    and   serial    bus
% \cite{shen2005} will server  as an intergral part of commercialisation  of quantum computing
% power.  Superconducting quibts are one of the device sof choice

\noindent Superconducting qubits are one of the promising trends for implementation of quantum
computing technololgy. Being  nothing more than strips  of aluminum on a  chip, their geometry
can be  designed to select  an operating energy, state  transition rates and  sensitivity that
would fit  the use-case of very  specific environments.  Over  the past decade they  have been
carried out  the functionality of a  transistor \cite{Astafiev2010}\cite{hoi2011}, multiplexer
\cite{honigl2018}  and serial  bus \cite{shen2005}.   Superconducting qubits  can be  produced
using industry  standard fabrication techniques  and integrated  at scale into  large coherent
circuits  \cite{johnson2010}.   All speaks  to  a  strong  case  of servicing  future  quantum
electronic platforms with this technology.

One of the inherent limitation superconducting  qubits face is a comparatively short coherence
time, $\tau_{\text{dec}}$.  It  is a time over  with the $\rho_{01}$ and $\rho_{10}$  components of the
system's                                    density                                    matrix,
$\rho    =    \ensuremath{\left(\begin{smallmatrix}    \rho_{00}    &    \rho_{01}    \\    \rho_{10}    &
      \rho_{11} \end{smallmatrix}\right)} $, decay to  zero and computational information becomes
lost \cite{phaseExp}.  For  example, the transmon qubits in the  revolutionary 5-qubit quantum
experience from IBM  (\href{http://www.research.ibm.com/ibm-q}{research.ibm.com/ibm-q}) have a
coherence  time   of  $  \sim  60\,\mu   $s  \cite{linke2017}.  In  contrast,   qubits  on  trapped
$  ^{+}  $Be$  ^{9}  $  ions  already  had  a  coherence  time  of  $\sim1\,  $ms  20  years  ago
\cite{monroe1995}.  A  strong decoherence is a  consequence of the large  capacitances used by
the  superconducting  qubits,  which  couples  them  to  charge  variations  in  the  external
environment: the  loop of the  flux qubits are typically  $ 1\iunitMixed{\mu}{m}$ or  greater in
size  \cite{Astafiev2010}\cite{hoi2011}\cite{johnson2010}).  The  charge fluctuations  in  the
environment result  in random  changes of  the qubit's  energy levels,  leading to  an erratic
evolution of the quantum state that `averages' out information on the off-diagonal elements in
the  density  matrix \cite{devoret2008}.   To  house  a  sufficient  number of  quantum  logic
operations for  multi-stage computations,  $ \sim  10^4 $,  coherence times  need to  surpass the
$ 100\,\mu$s barrier \cite{orlando1999}.
 
Flux qubit architectures have been developed to address this decoherence problem by making the
energy of their  Josephson junctions (JJ) dominant  in the system, which  lowered the device's
charge  sensitivity  \cite{orlando1999}  \cite{chiorescu2003} \cite{mooij1999}.   The  further
branching  of  flux   qubit  designs  has  lead  to  improved   coherence  times:  quantronium
$\sim500\,$ns \cite{cottet2002} \cite{gu2017}, shunted phase  qubit $\sim10\,\mu $s \cite{stern2014} ,
shunted  flux  qubit  $\sim80\,\mu$s  \cite{yan2016}  ,  4-JJ  \cite{qui2016},  fluxonium  $\sim1\,$ms
\cite{pop2014}.
 
This list is extended  with a `twin' qubit, consisting of two  symmetrical flux qubits, linked
by a  common $ \alpha-$Josephson  Junction (Fig.~\ref{fig:setup}).  A chain  of 15 such  qubits was
recently  placed  into  a  coplanar  waveguide to  demonstrate  flux-tunable  transmission  of
microwaves \cite{shulga2018}.   Of particular interest to  us was the weak  flux dependence of
the chain's transition energy when it was  biased to the degeneracy point $\frac{1}{2}\Phi_0 $
pointing  to  an  operational regime  which  would  benefit  from  both low  flux  and  charge
sensitivities.
 
In this work we isolate one of these twin qubits and provide experimental evidence for: strong
anharmonicity with respect to the \iket{1}\ilra\iket{2} and \iket{2}\ilra\iket{3} transitions;
weak   flux    dependence   of    the   transition   energies    at   the    degeneracy   bias
$\sim \frac{1}{2}\Phi_0 =  \frac{h}{4e}$; compliance with simulations of  the energy structure
and controllability of the \iket{1}~\ilra~\iket{2} transition rate.

% tuneable capacitive coupling to the tranmission  line, resulting from a flux-tuneable dipole
% moment.
