% -*- TeX-master: "../dipole_ilya_paper.tex" -*-
\section{Fabrication}
% \red{natural  to couple  capacitively} \red{Thickness  of
% aluminium size of chip }

% \red{Glued to a pcb and bonded with gold wires, connectio
% input and output to coax lines.}

\begin{figure}[h]
  \includegraphics[height=10.5cm]{figure1_v3}
  \caption{\small  \textbf{Geometry of  a twin  qubit.}  a)
    Scanning electron  microscope image  of the  qubit flux
    with  phase biases  $  \varphi  $, $  \eta\varphi  $  applied to  it's
    superconducting loops. The repeated line structures are
    the  byproduct  of   the  double-angle  evaporation  of
    aluminum that creates the Al-AlO$_x$-Al JJs highlighted
    in red and pink.  Each of  the qubits is coupled to the
    transmission line with a T-shaped capacitor (inset); b)
    The  twin qubit  is  a symmetrical  arrangement of  two
    individual    flux    qubits     (as    described    in
    \cite{orlando1999})  sharing the  central JJ.   Islands
    are labeled with a  Cooper pair occupation n$_i$, phase
    $\varphi_i$  and voltage  V$_i$,  with the  ground setting  a
    reference of  0 for  all three variables.   JJs (marked
    with a  cross) mediate  capacitive and  JJ interactions
    between  the  islands.   The  central  junction  has  a
    capacitance  of  $\alpha$C  and  an  energy  $  \alpha$E$_{J}$  ,
    compared with the C$_J$, E$_J$ values of the others.}
  \label{fig:setup}
\end{figure}

%% Describe fabrication
\noindent The  qubits and transmission line  are fabricated
on an undoped 100 silicon substrate, which is pre-patterned
with   10\,nm   NiCr\,-\,90\,nm   Au  ground   planes   and
markers. We begin by cleaning  the wafer for $\sim$~10 minutes
at 60\,C in acetone rinsing in de-ionized water. Two layers
of  electron resist  are sequentially  spun and  post baked
(3\,minutes  at 60\,C)  onto the  wafer: {Copolymer  13\%},
700\,nm; ZEP520a:Anisol  2:1, 60\,nm.   We use  an electron
beam  lithographer to  expose  the resist  using a  30\,kV,
10\,pA  beam delivering  a dose  of $  70\,\mu $C/cm$^2$.  We
develop the pattern in Pylxylen for 35\,seconds followed by
a 5\,minute submersion in Isopropanol:H$_2$0 93:7 and rinse
in Isopropanol. Shadow evaporation of aluminum in a Plassys
\cite{wu2013}   simultaneously   deposits    the   JJ   and
transmission  line  structures.  We  first  use  \red{argon
  etching}  to  remove  oxide   layers  for  good  galvanic
contact, and, maintaining high vacuum, we deposit 20\,nm of
Al and perform static oxidation for 10 minutes at 0.3\,mBar
to generate the intermediate AlO$_x$ insulating barrier for
the JJ.  A second 30\,nm layer of Al completes the process.

These steps give  us the 5-JJ structure of  the twin qubits
and T-shaped  capacitors coupling them to  the transmission
line, see Fig.~\ref{fig:setup}.   Each JJ has an  a area of
\iunit{400\times200}{nm$^2$}.   The  coplanar transmission  line
with  impedance $  Z_{0}  \sim  50\,\Omega $  runs  to the  opening
between the ground planes in the center of the chip.


We  bond the  sample chip  to a  printed circuit  board and
mount  it on  a holder  with a  niobium-coil magnet  on the
13\,mK stage of  a dilution refrigerator. A  metal cover is
used to shield the holder from stray magnetic fields, while
attenuators,  -50\,dBm on  the 50K  stage, -30\,dBm  on the
4\,K stage,  thermalize the  input lines to  the respective
temperatures.   We attach  a 4\,K  stage circulator  on the
output  line  for  isolation,  and amplify  the  signal  by
+35\,dBm on the 4K stage  and +35\,dBm at room temperature.
This system of attenuators  and amplifiers facilitate power
conversion  between  the  laboratory  equipment  and  qubit
microwaves.  We  surround  each temperature  stage  of  the
system with a  case, wrapping it like a  Russian doll, that
thermally shield  from higher stages.  Prior  to performing
characterization   measurement,  we   took  the   microwave
transmission spectrum with the qubit detuned, and corrected
all  further measurements  by this  background transmission
profile.

Our pilot experiment  did not to go to the  depths of using
chemical and physical treatment of the substrate surface to
remove two-level system defects  in the silicon oxide layer
\cite{earnest2018}  or   employing  infra-red   filters  to
eliminate      stray      light     during      measurement
\cite{barends2011}.  We plan on doing these improvements in
further development cycles of the twin qubit.

%%% Local Variables:
%%% mode: latex
%%% TeX-master: "../dipole_ilya_paper"
%%% End:
