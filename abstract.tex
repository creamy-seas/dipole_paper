% -*- TeX-master: "../dipole_ilya_paper.tex" -*-

\begin{abstract}
  \noindent Short decoherence times stand as one of the main obstacles hindering industrial scale implementation of quantum processors based on superconducting qubits. The simplicity and flexibility of fabricating superconducting qubits compared with other qubit architectures, most notably trapped ions, has to be set against their diminished ability of handling sequential state operations - the most successful processor prototype has a decoherence time of $\sim60 \iunitMixed{\mu}{s} $ and a gate operation time of $\sim20 \iunitMixed{\mu}{s}$ \cite{linke2017}, so within 3 gate operations it's quantum state becomes substantially deteriorated.
  
  Right now, in the embryonic stages of quantum infrastructure development, probability theory of the multi-armed bandit dictates \cite{gittins1989}: to maximize expected gains in combating this decoherence problem, resources should be allocated to exploring new alternatives, as opposed to exploiting the possibly sub-optimal (requiring a supporting qubit overhead that is double the size of the original system) error correction mechanism already in place \cite{reed2011}.
  
  In this work we investigate alternative of boosting the qubit's decoherence time using a new geometry - a fusion of two flux qubits joined by a common Josephson Junction. At the degeneracy flux-bias point, $ \frac{1}{2}\Phi_0 $, the twin qubit is measured to have energy spectrum plateaus with curvature several orders of magnitude lower than that in previous flux qubits. This flatness makes the qubit more robust to flux noise, and under the most basic of fabrication cycles, the qubit demonstrates a decoherence time of 42\,ns. We experimentally show the transmission spectrum, Rabi oscillations and simulate the dipole transition on this new type of qubit, laying the out the initial characteristics for future developments.
\end{abstract}

%%% Local Variables:
%%% mode: latex
%%% TeX-master: "../dipole_ilya_paper"
%%% End:
